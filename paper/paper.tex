\documentclass[11pt,a4paper]{article}
\usepackage[english]{babel}
\usepackage[square,numbers]{natbib}
\bibliographystyle{abbrvnat}
\usepackage[margin=1in]{geometry}

\usepackage{amsmath,amssymb,amsthm}
\usepackage{hyperref}
\usepackage{pgf, tikz}
\usepackage{xspace, units}
\usepackage{typearea}
\allowdisplaybreaks

\usepackage{lmodern}
\usepackage[T1]{fontenc}
\usepackage[utf8]{inputenc}
\usepackage{textcomp}

\newtheorem{definition}{Definition}
\newtheorem{theorem}{Theorem}
\newtheorem{proposition}[theorem]{Proposition}
\newtheorem{lemma}[theorem]{Lemma}
\newtheorem{corollary}[theorem]{Corollary}
\newtheorem{fact}[theorem]{Fact}
\newtheorem{claim}[theorem]{Claim}
\newtheorem{observation}[theorem]{Observation}

\newcommand{\NN}{\ensuremath{\mathbb{N}}}
\newcommand{\ZZ}{\ensuremath{\mathbb{Z}}}
\newcommand{\RR}{\ensuremath{\mathbb{R}}}

\renewcommand{\Pr}[1]{\mbox{\rm\bf Pr}\left[#1\right]}
\newcommand{\Ex}[2][]{\mbox{\rm\bf E}_{#1}\left[#2\right]}
\newcommand{\1}[1]{\mbox{\rm\bf 1}_{#1}}

\author{Ayk Borstelmann}

\title{Revenue Maximization for Buyers with Costly Participation}

\date{02.03.2025}

\begin{document}

\maketitle

\begin{abstract}
    Test
\end{abstract}

\begin{section}{Introduction}
 Commonly in mechanism design problems auctions are modeled as free to participate in.
 However, this does not necessarily reflect how in reality mechanism exist.
 For example one might need to invest travel costs to arrive to an auction or incur some cost of opportunity.
 All those costs we refer to as participation costs of an auction.
 As buyers might not be convinced to share those costs publically beforehand, e.g. since it is sensitive information, we are studying the problem of private participation costs.

 Recall the setting of the single item auction.
 There is a set $\mathcal{N}$ of $n$ buyers that have interest in buying an item.
 Each buyer has a private valuation $v_i \in \RR_{\geq 0}$.
 A mechanism is a tuple $\mathcal{M} = (x,p)$ of an allocation function $x: \RR \rightarrow [0,1]^n$ and a payment function $p: \RR \rightarrow \RR^n$.
 A players utility with respect to that mechanism is defined as $u_i^\mathcal{M}(v_i) = v_i x_i(v_i) - p_i(v_i)$.

 To model the private participation costs, each player now additionally has participation costs $c_i \in \RR$ \cite{primary}.
 The mechanism $\mathcal{M}' = (x', p')$ will now respect that participation costs as well.
 An player's utility is now reduced by her participation cost if she participates.

 \begin{align*}
     u_i^{\mathcal{M}'} = \begin{cases}
                              v_i x_i(v_i, c_i) - p_i(v_i, c_i) - c_i & \text{if $i$ participates} \\
                              0                                       & \text{otherwise}
                          \end{cases}
 \end{align*}

 We will now call a player's type the tuple of valuation and participation cost $t_i = (v_i, c_i)$.
 We study the setting where each player's type is drawn from a distribution $\bar{F_i}$.
\end{section}

\begin{section}{Single Buyer Mechanism Design}

 \subsection{Incentive Compatible Mechanism \& Revenue}
 In auction design one usually tries to find incentive compatible mechanism and then to maximize the revenue under the assumptions that players reveal their type truthfully.

 For the first part, in usually single parameter mechanisms \cite{myerson} found a clear characterization of a truthful mechanism.

 \begin{lemma}[\cite{myerson}]
     For single parameter environments, an allocation rule $x$ is implementable if and only if it is monotone.
     If $x$ is monotone, then there exists a unique payment rule $p$, s.t. the mechanism $M=(x,p)$ is truthful
     and $p$ is given by:
     \begin{align*}
         p(v) = v x(v) - \int_0^v x(t) dt + p_0
     \end{align*}
 \end{lemma}

 Also, \cite{primary} came up with a characterization of mechanism which are enough to focus on.

 \begin{lemma}{\cite{primary}}
     \label{lemma:thruthful-mechanism}
     It is without loss for seller to commit to a truthful single parameter mechanism $\mathcal{M}=(x,p)$ and
     for the buyer to participate and truthfully reveal v if and only if $u^\mathcal{M}(v) = \int_0^v x(z)dz - p_0 \geq c$.
 \end{lemma}

 Next to maximize the mechanism one has to find its characteristics as well.
 In the single parameter setting, \cite{myerson} provided such as well.

 \begin{lemma}[\cite{myerson}, Myerson Revenue]
     \label{lemma:myerson-revenue}

     Let $\mathcal{M}=(x,p)$ be a truthful single-parameter mechanism, then the expected revenue equals the expected virtual welfare. That is
     \begin{align*}
         \mathbf{E}_v\left[p(v)\right]
         = \mathbf{E}_v\left[x(v)\varphi(v)\right] + p_0
     \end{align*}

     where $\varphi(t) = t - \frac{1 - F(t)}{f(t)}$
 \end{lemma}

 Following from that we derive an expression for the expected revenue for private participation costs.
 For that we first formulize the mechanism described in \autoref{lemma:thruthful-mechanism}.

 Given a truthful single parameter mechanism $\mathcal{M}=(x,p)$.
 For that, we define $v_x(c)$ as the minimum value a player must have, s.t. she would still participate in the mechanism $\mathcal{M}$ when having participation cost $c$.

 \begin{align*}
     v_x(c) = \inf_{v \in \RR_{\geq 0}} \{vx(v) - p(v) \geq c\}
 \end{align*}

 Next we can define the allocation function $x_c$ as simply implementing the characterization of \autoref{lemma:thruthful-mechanism},
 by allocating $x(v)$ if the player has a value that is at least $v_x(c)$ and $0$ otherwise.

 \begin{align*}
     x_c(v) = \begin{cases}
                  x(v) & v \geq v_x(c)    \\
                  0    & \text{otherwise}
              \end{cases}
 \end{align*}

 Similar the payment rule $p_c(v)$ delegates the payment rule $p$ in the case that $v \geq v_x(c)$ and otherwise expects no payment.
 Together, they define the mechanism $\mathcal{M}_c = (x_c, p_c)$ induced by the mechanism $\mathcal{M}$.

 If we now define a virtual value function that is conditional on $c$, by using the conditional cumulative distribution $\bar{F}_c$ and the conditional density function $\hat{f}_c$ as
 $\varphi_c(v) = v - (1- \bar{F}_c(v)) / \hat{f}_c(v)$ then we can express the expected revenue in \autoref{theorem:expected-revenue}.

 \begin{theorem}[\cite{primary}]
     \label{theorem:expected-revenue}
     Given any $\bar{F}$ with marginal cost distribution $G$, any mechanism $\mathcal{M}=(x,p)$, then the expected revenue $\mathbf{E}_{t \sim \bar{F}}\left[p_c(v)\right]$ of the seller is

     \begin{align*}
         \mathbf{E}_{c \sim G}\left[\mathbf{E}_{v\sim\bar{F}_c}\left[x_c(v)\varphi_c(v)\right] - (1-\bar{F}_c(v_x(c))) \cdot \max\{-p_0,c\}\right]
     \end{align*}
 \end{theorem}

 \begin{proof}
     We proof this using a case distinction and then draw the expectation.
     First start by fixing $c$ arbitrarily.
     Consider the case that $c > -p_0$. Note that $-p_0$ corresponds to a payment the player would get from the mechanism for participating.
     That means in the case that $c > -p_0$, we do not simply always participate, since our utility could be negative. Therefore, we also have $v_x(c) > 0$.

     We first consider the case that we participate, i.e. $v \geq v_x(c)$.
     In that case the payment the player has to pay the payment $p(v)$ of the single parameter mechanism.

     \begin{align}
         \label{eq:payment-participating}
         p_c(v) = p(v) = v x(v) - \int_0^v x(t) dt + p_0
     \end{align}

     We use linearity of the integral and the definition of utility to express this in terms of the allocation function $x_c$.
     \begin{align*}
         \int_0^v x(t) dt & = \int_0^{v_c(x)} x(t) dt + \int_{v_c(x)}^v x(t) dt     \\
                          & = u^\mathcal{M}(v_c(x)) + p_0 + \int_{v_c(x)}^v x(t) dt \\
                          & = c + p_0 + \int_{v_c(x)}^v x(t) dt
     \end{align*}
     Here the last step uses that $u^\mathcal{M}(v_c(x)) = c$ per definition of $v_c(x)$.

     Plugging this into \autoref{eq:payment-participating} yields
     \begin{align*}
         p_c(v) = v x_c(v) - \int_0^v x_c(t) dt - c
     \end{align*}

     Next we consider the case that the player does not participate, i.e. $v < v_x(c)$.
     In that case, the payment rule requests no payment, i.e. $p_c(v) = 0$, yet conveniently some part of the expression found before are also $0$:

     \begin{align*}
         p_c(v) = 0 = v x_c(v) - \int_0^v x_c(t) dt
     \end{align*}

     Thus, we can represent this entire case using an indicator function:
     \begin{align*}
         p_c(v) = 0 = v x_c(v) - \int_0^v x_c(t) dt - c \1{v \geq v_x(c)}
     \end{align*}

     Drawing the expectation around the value $v \sim \bar{F}_c$ while keeping the participation cost still fixed yields

     \begin{align*}
         \Ex[v \sim \bar{F}_c]{p_c(v)} & = \Ex[v \sim \bar{F}_c]{v x_c(v) - \int_0^v x_c(t) dt - c \1{v \geq v_x(c)}}   \\
                                       & = \Ex[v \sim \bar{F}_c]{v x_c(v) - \int_0^v x_c(t) dt} - \Pr{v \geq v_x(c)}c   \\
                                       & = \Ex[v \sim \bar{F}_c]{x_c(v) \varphi_c(v)} - (1 - \bar{F}_c(v_x(c))) \cdot c \\
     \end{align*}

     Here in the last step we used that for fixed $c$, $M_c$ is a truthful mechanism thus we can apply \autoref{lemma:myerson-revenue}, and we also used the cumulative distribution.

     Now we consider the opposite case, that is that $c \leq -p_0$.
     Analogously to the case before, $c \leq -p_0$ means that the mechanism's payment for simply participating already pays of our participation cost.
     Certainly in that case $v_x(c) = 0$.

     Therefore, the player always participates and pays $p(v)$. In the case that $v_x(c) = 0$, we have $x_c(v) = x(v)$ for all $v \in \RR_{\geq 0}$, therefore:
     \begin{align*}
         p_c(v) = p(v) = v x(v) - \int_0^v x(t)dt + p_0 = v x_c(v) - \int_0^v x_c(t)dt + p_0
     \end{align*}

     Drawing again the expectation around the value $v \sim \bar{F}_c$ while keeping the participation cost still fixed yields


     \begin{align*}
         \Ex[v \sim \bar{F}_c]{p_c(v)} & = \Ex[v \sim \bar{F}_c]{v x_c(v) - \int_0^v x_c(t)dt + p_0}                  \\
                                       & = \Ex[v \sim \bar{F}_c]{x_c(v) \varphi_c(v)} + p_0                           \\
                                       & = \Ex[v \sim \bar{F}_c]{x_c(v) \varphi_c(v)} + (1 - \bar{F}_{c}(v_x(c))) p_0
     \end{align*}

     Where in the last equality, we used that since $v_x(c) = 0$, $\bar{F}_{c}(v_x(c)) = 0$ as well.

     Now we can finally draw the expectation around the previously fixed $c$ and get to the wished theorem:

     \begin{align*}
         \Ex{p_c(v)} & = \Ex[c \sim G]{\Ex[v \sim \bar{F}_c]{x_c(v) \varphi_c(v)} + (1 - \bar{F}_{c}(v_x(c))) \max\{c, -p_0\} }
     \end{align*}
 \end{proof}

 \subsection{Revenue Maximization}

 We observe that this problem is inherently non-convex. Even more, \cite{primary} showed that it is not even reparameterizedable as a convex problem.
 Due to that we cannot hope to calculate revenue optimal mechanisms in general in polynomial time.

 Therefore, \cite{primary} propose to introduce a fully polynomial-time approximation scheme (FPTAS).

 First we will normalize $v$ to $[0,1]$.
 Next, let $\epsilon \in (0, 1)$ be the approximation degree.
 Let $\bar{F}'$ be the value distribution with the value rounded down to the nearest multiple of $\epsilon$.
 Since still there are infinitely many allocation functions, we discretize those two by only allocating multiples of $\epsilon$ as well.

 This leads us to a problem definition for which an optimal mechanism is computable in polynomial time w.r.t. to $\frac{1}{\epsilon}$.
 We will see later how to do this.
 Still it is not clear that this mechanism is close to the optimal mechanism for the original problem.

 \begin{theorem}
     \label{theorem:approximation-guarantee}
     Let $\mathrm{Rev}(\bar{F}, \mathcal{M})$ be the revenue of mechanism $\mathcal{M}$ on type distribution $\bar{F}$.
     For any type distribution $\bar{F}$ supported on $[0,1]^2$, for any $\epsilon \in (0,1)$,
     the optimal mechanism $\mathcal{M}'$ on $\bar{F}'$ that only allocates multiples of $\epsilon$, achieves an expected revenue $\mathrm{Rev}(\bar{F},\mathcal{M}')$ of at least
     $\mathrm{OPT}(\bar{F}) - 4\epsilon$, where $\mathrm{OPT}(\bar{F})$ refers to the optimal expected revenue on $\bar{F}$.
 \end{theorem}

 To proof this we make use of the following lemmas without proofs:
 \begin{lemma}[\citet{primary}]
     \label{lemma:allocation-function-within-e}
     For any distribution $\bar{F}$ supported on $[0,1]^2$ and for any pair of mechanisms $\mathcal{M}'$ and $\widehat{\mathcal{M}}$
     with allocation rules $x'$ and $\hat{x}$, s.t. $x'(v) \in [\hat{x}(v), \hat{x}(v) + \epsilon]$ for all $\epsilon$, we have $\mathrm{Rev}(\bar{F}, \mathcal{M}') \geq \mathrm{Rev}(\bar{F}, \widehat{\mathcal{M}}) - \epsilon$.
 \end{lemma}
 \begin{lemma}[\citet{primary}]
     \label{lemma:difference-in-optimal-mechanims}
     Let $(\Omega, \mathcal{F}, P)$ be any probability measure and let $t_1, t_2: \Omega \rightarrow \mathbb{R}^2$ be two 2-dimensional random variables.
     If $\sup_{\omega \in \Omega} || t_1(\omega) - t_2(\omega) ||_\infty \leq \epsilon$, then $|\mathrm{OPT}(t_1) - \mathrm{OPT}(t_2)| \leq 3\epsilon$.
 \end{lemma}

 \begin{proof}[Proof of \autoref{theorem:approximation-guarantee}.]
     For any $\bar{F}$, let $\mathcal{M}'$ be the optimal mechanism on the discretized type distribution $\bar{F}'$ for which the allocation rule is also discretized.
     First, since we defined $\bar{F}'$ as the \textit{rounded down} type distribution, for any mechanism $\mathcal{M}$ holds, that $\mathrm{Rev}(\bar{F}, \mathcal{M}) \geq \mathrm{Rev}(\bar{F}', \mathcal{M})$, so in particular this also applies to $\mathcal{M}'$.

     Next, we define $\widehat{\mathcal{M}}$ as the optimal mechanism on $\bar{F}'$ (one that is allowed to allocate continuous values).
     Next, we make use of \autoref{lemma:allocation-function-within-e}, since our allocation function $x'$ of mechanism $\mathcal{M}'$
     allocates within an $\epsilon$-interval of the allocation function $\hat{x}$ of $\widehat{\mathcal{M}}$.
     Thus, according to \autoref{lemma:allocation-function-within-e}, we got $\mathrm{Rev}(\bar{F}', \mathcal{M}') \geq \mathrm{Rev}(\bar{F}', \widehat{\mathcal{M}}) - \epsilon$.

     Finally, note, that for the two type distributions $\bar{F}$ and $\bar{F}'$ it holds that they originate from the same probability measure, yet the value of the random variables differs by at most $\epsilon$.
     Also note, that we did not round the participation cost, thus the difference with respect to the infinity-norm $|| \cdot ||_{\infty}$ is also at most $\epsilon$.
     Therefore, we can apply \autoref{lemma:difference-in-optimal-mechanims}, since $\widehat{\mathcal{M}}$ is optimal mechanism on $\bar{F}'$,
     we get $\mathrm{Rev}(\bar{F}', \widetilde{\mathcal{M}}) \geq \mathrm{OPT}(\bar{F}') - 3\epsilon$.

     If we put now all together, we get:
     \begin{align*}
         \mathrm{Rev}(\bar{F}, \mathcal{M}') & \geq \mathrm{Rev}(\bar{F}', \mathcal{M}')                     \\
                                             & \geq \mathrm{Rev}(\bar{F}', \widehat{\mathcal{M}}) - \epsilon \\
                                             & \geq \mathrm{OPT}(\bar{F}') - 4\epsilon
     \end{align*}
 \end{proof}

 Now that we know that the difference in the problem statement leads to an reasonable approximation guarantee,
 we can now dedicate to finding an optimal mechanism for this problem.

 We will use a dynamic program based on the approximation degree $\epsilon$ to calculate for all possible valuations
 an allocation function that optimizes the revenue.
 We will make sure this allocation rule is non-decreasing and will follow the payment rule through \autoref{lemma:thruthful-mechanism}
 in order to also get a truthful mechanism.

 For $i,j \leq \frac{1}{\epsilon}$ and $k \leq \frac{1}{\epsilon^2}$,
 we define $R(i,j,k)$ as the optimal revenue from only a subset of the buyers, i.e. those that have a value $v \leq i \epsilon$
 and for which the allocation function is fixed to $x(i \epsilon) = j \epsilon$ and the utility is also fixed to $u(i \epsilon) = k \epsilon^2$.
 Observe that in this formulation, we can express it in terms of an expectation:

 \begin{equation}
     \label{eq:rijk}
     R(i,j,k) = \max_{\mathcal{M}: x(i\epsilon) = j\epsilon, u(i\epsilon) = k \epsilon^2} \Pr{v \leq i\epsilon}\Ex{p_c(v)\mid v \leq i \epsilon}
 \end{equation}

 \begin{theorem}
     \label{theorem:recursion-formular}
     For $i \geq 2$ $R(i,j,k)$ can be calculated as follows:

     \begin{align*}
         R(i,j,k) = \max_{j' \leq j} R(i-1, j, k - j) + q_i \bar{F}_i'(k \epsilon^2) \cdot (i\cdot j - k)\epsilon^2
     \end{align*}

     where $q_i$ refers to the occurrence probability of $v = i \epsilon$ and $\bar{F}_i'$ refers to the cumulative type distribution conditional on the value being $i\epsilon$.
 \end{theorem}

 From this we can directly deduce the following corollary.

 \begin{corollary}
     The optimal mechanism $\mathcal{M}'$ that only allocated multiple of $\epsilon$ on any discretized type distribution $\bar{F}'$
     can be computed in $\mathcal{O}\left(\frac{1}{\epsilon^5}\right)$.
 \end{corollary}
 \begin{proof}
     Observe that from the definition of $R(i,j,k)$, the following statement holds:
     \begin{align*}
         \mathrm{Rev}(\bar{F}', \mathcal{M}') = \max_{j \leq \frac{1}{\epsilon}, k \leq \frac{1}{\epsilon^2}} R\left(\frac{1}{\epsilon}, j, k\right)
     \end{align*}

     Using dynamic programming by applying \autoref{theorem:recursion-formular} and fixing the base cases $R(1,j,k) = 0$ for any $k \leq j$ and $R(1,j,k)=-\infty$ for any $k > j$,
     we can calculate for all $\frac{1}{\epsilon^4}$ combinations of $i,j$ and $k$ the respective values of $R(i,j,k)$.
     Each of those $R(i,j,k)$ needs at most $\frac{1}{\epsilon}$ calculations because of the maximum.
     Therefore, we calculate the revenue of $\mathcal{M}'$ in $\mathcal{O}\left(\frac{1}{\epsilon^5}\right)$ time.
     Recovering the actual mechanism $\mathcal{M}'$ and its allocation rule from that is simply a matter of backtracking the $j'$ that maximized our expression.
 \end{proof}

 All that is left is to proof \autoref{theorem:recursion-formular}.
 \begin{proof}[Proof of \autoref{theorem:recursion-formular}]
     First recall \autoref{eq:rijk}:

     \begin{align*}
         R(i,j,k) = \max_{\mathcal{M}: x(i\epsilon) = j\epsilon, u(i\epsilon) = k \epsilon^2} \Pr{v \leq i\epsilon}\Ex{p_c(v)\mid v \leq i \epsilon}
     \end{align*}

     Also, observe that we can use the linearity of expectation to split up the expectation of \autoref{eq:rijk} into the case that $v \leq (i-1)\epsilon$ and the case that $v = \epsilon$.
     \begin{align*}
           & \Ex{p_c(v)\mid v \leq i \epsilon}                                                                                                                                      \\
         = & \Pr{v \leq (i-1)\epsilon \mid v \leq i \epsilon} \Ex{p_c(v) \mid v \leq (i - 1) \epsilon} + \Pr{v = i \epsilon \mid v \leq i \epsilon} \Ex{p_c(v) \mid v = i \epsilon} \\
     \end{align*}

     Next if we pull in the $\Pr{v \leq i\epsilon}$ it neatly cancels out the condition on the probabilities, leading to:

     \begin{align*}
           & R(i,j,k)                                                                                                                                                                                             \\
         = & \max_{\mathcal{M}: x(i\epsilon) = j\epsilon, u(i\epsilon) = k \epsilon^2} (\Pr{v \leq (i-1)\epsilon} \Ex{p_c(v) \mid v \leq (i - 1) \epsilon} + \Pr{v = i \epsilon} \Ex{p_c(v) \mid v = i \epsilon})
     \end{align*}

     Further, observe that for $p_c(v)$ in the second expectation, everything is already fixed through during the maximization, i.e. for the fixed value $v = i \epsilon$ the allocation is fixed and the utility is fixed, thus also the payment.
     This means, we can pull out this term, since the maximization does not affect it.
     So, we can focus on the term $\max_{\mathcal{M}: x(i\epsilon) = j\epsilon, u(i\epsilon)=k\epsilon^2} \Pr{v \leq (i-1)\epsilon} \Ex{p_c(v) \mid v \leq (i - 1) \epsilon}$ on its own.

     Note that this is similar to \autoref{eq:rijk} with $v = (i-1) \epsilon$, only the maximization constraints differ.
     For that we only need to fix $x((i-1)\epsilon)$ and $u((i-1)\epsilon)$.

     Conveniently, we can express the latter in terms of things that are already fixed.
     Note that we can write $u(i\epsilon)$ as follows:

     \begin{align*}
         u(i\epsilon) & = \int_0^{i\epsilon} x(t) dt                                             \\
                      & = \int_0^{(i-1)\epsilon} x(t) dt + \int_{i-1\epsilon}^{i\epsilon} x(t)dt \\
                      & = u((i-1)\epsilon) + \epsilon j \epsilon^2                               \\
     \end{align*}
     Where in the last the step we used that $x(i\epsilon) = j\epsilon$.

     Rearranging this, yields
     \begin{align*}
         u((i-1)\epsilon) = u(i\epsilon) - j \epsilon^2 = (k - j) \epsilon^2
     \end{align*}

     For $x((i - 1)\epsilon)$ one cannot find a direct solution derived from the constraints of $x(i\epsilon)$, thus we apply an actual maximization here.
     However, we must ensure that $x$ is non-decreasing, thus $j' \epsilon = x((i-1)\epsilon) \leq x(i\epsilon) = j\epsilon$.

     Combining this yields the following expression for the first term:
     \begin{align*}
         \max_{\mathcal{M}: x(i\epsilon) = j\epsilon, u(i\epsilon) = k \epsilon^2} \Pr{v \leq (i-1)\epsilon} \Ex{p_c(v) \mid v \leq (i - 1) \epsilon} = \max_{j' \leq j} R(i-1, j, k - j)
     \end{align*}

     Next we focus on the second term $\Pr{v = i \epsilon} \Ex{p_c(v) \mid v = i \epsilon}$ given that $x(i\epsilon) = j\epsilon$ and $u(i\epsilon) = k\epsilon^2$.
     $\Pr{v = i \epsilon} = q_i$ per definition of $q_i$.
     The expected revenue given that $v = i \epsilon$ depends on the event that the buyer participates.
     She only does this, if her participation cost is higher than the utility gain from the single-parameter mechanism.
     In order words:

     \begin{align*}
          & \Ex{p_c(v) \mid v = i \epsilon}_{x(i\epsilon) = j \epsilon, u(i\epsilon) = k\epsilon^2}                                         \\
          & = \Pr{c \geq u(i\epsilon)} \Ex{p_c(i\epsilon) \mid c \leq u(i\epsilon)}_{x(i\epsilon) = j \epsilon, u(i\epsilon) = k\epsilon^2} \\
          & = \bar{F}_i'(k\epsilon^2) \Ex{p(i\epsilon)}_{x(i\epsilon) = j \epsilon, u(i\epsilon) = k\epsilon^2}                             \\
          & = \bar{F}_i'(k\epsilon^2) \Ex{u(i\epsilon) - i\epsilon \cdot x(i\epsilon)}                                                      \\
          & = \bar{F}_i'(k\epsilon^2) (k - i \cdot j)\epsilon^2                                                                             \\
     \end{align*}

     This yields for the second term:
     \begin{align*}
           & \Pr{v = i \epsilon} \Ex{p_c(v) \mid v = i \epsilon}_{x(i\epsilon) = j \epsilon, u(i\epsilon) = k\epsilon^2} \\
         = & q_i \bar{F}_i'(k\epsilon^2) (k - i \cdot j)\epsilon^2
     \end{align*}

     Combining both terms then results in our goal:
     \begin{align*}
         R(i,j,k) = \max_{j' \leq j} R(i-1, j, k - j) + q_i \bar{F}_i'(k \epsilon^2) \cdot (i\cdot j - k)\epsilon^2
     \end{align*}
 \end{proof}

\end{section}

\bibliography{references} % Insert name of .bib file

\end{document}
