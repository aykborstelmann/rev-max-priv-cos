\documentclass[11pt,a4paper]{article}
\usepackage[margin=1in]{geometry}

\usepackage{amsmath,amssymb,amsthm}
\usepackage{hyperref}
\usepackage{pgf, tikz}
\usepackage{xspace, units}
\usepackage{typearea}
\allowdisplaybreaks

\usepackage{lmodern}
\usepackage[T1]{fontenc}
\usepackage[utf8]{inputenc}
\usepackage{textcomp}

\newtheorem{definition}{Definition}
\newtheorem{theorem}{Theorem}
\newtheorem{proposition}[theorem]{Proposition}
\newtheorem{lemma}[theorem]{Lemma}
\newtheorem{corollary}[theorem]{Corollary}
\newtheorem{fact}[theorem]{Fact}
\newtheorem{claim}[theorem]{Claim}
\newtheorem{observation}[theorem]{Observation}

\newcommand{\NN}{\ensuremath{\mathbb{N}}}
\newcommand{\ZZ}{\ensuremath{\mathbb{Z}}}
\newcommand{\RR}{\ensuremath{\mathbb{R}}}

\renewcommand{\Pr}[1]{\mbox{\rm\bf Pr}\left[#1\right]}
\newcommand{\Ex}[2][]{\mbox{\rm\bf E}_{#1}\left[#2\right]}
\newcommand{\1}[1]{\mbox{\rm\bf 1}_{#1}}

\author{Ayk Borstelmann}

\title{Revenue Maximization for Buyers with Costly Participation}

\date{02.03.2025}

\begin{document}

\maketitle

\begin{abstract}
    Test
\end{abstract}

\begin{section}{Introduction}
 Commonly in mechinism design problems auctions are modeled as free to participate in.
 However this does not necessarily reflect how in reality mechanism exist.
 For example one might need to invest travel costs to arrive to an aution or incur some cost of opportunity.
 All those costs we refer to as participation costs of an auction.
 As buyers might not be convinced to share those costs publically beforehand, e.g. since it is a sensitive information, we are studying the problem of private participation costs.

 Recall the setting of the single item auction.
 There is a set $\mathcal{N}$ of $n$ buyers that have interest in buying an item.
 Each buyer has a private valuation $v_i \in \RR_{\geq 0}$.
 A mechanism is a tuple $\mathcal{M} = (x,p)$ of an allocation function $x: \RR \rightarrow [0,1]^n$ and a payment function $p: \RR \rightarrow \RR^n$.
 A players utility with respect to that mechanism is defined as $u_i^\mathcal{M}(v_i) = v_i x_i(v_i) - p_i(v_i)$.

 To model the private participation costs, each player now additionally has participation costs $c_i \in \RR$ \cite{primary}.
 The mechanism $\mathcal{M}' = (x', p')$ will now respect that participation costs as well.
 An player's utility is now reduced by her participation cost if she participates.

 \begin{align*}
     u_i^{\mathcal{M}'} = \begin{cases}
                              v_i x_i(v_i, c_i) - p_i(v_i, c_i) - c_i & \text{if $i$ participates} \\
                              0                                       & \text{otherwise}
                          \end{cases}
 \end{align*}

 We will now call a player's type the tuple of valuation and participation cost $t_i = (v_i, c_i)$.
 We study the setting where each player's type is drawn from a distribution $\bar{F_i}$.
\end{section}

\begin{section}{}
 In auction design one usually tries to find incentive compatible mechanism and then to maximize the revenue under the assumptions that players reveil their type thruthfully.

 For the first part, in usually single parameter mechanisms \cite{myerson} found a clear characterization of a thruthful mechanism.

 \begin{lemma}[\cite{myerson}]
     For single parameter environments, an allocation rule $x$ is implementable if and only if it is monotone.
     If $x$ is monotone, then there exists a unique payment rule $p$, s.t. the mechanism $M=(x,p)$ is truthful
     and $p$ is given by:
     \begin{align*}
         p(v) = v x(v) - \int_0^v x(t) dt + p_0
     \end{align*}
 \end{lemma}

 Also \cite{primary} came up with a characterization of mechanism which are enough to focus on.

 \begin{lemma}{\cite{primary}}
     \label{lemma:thruthful-mechanism}
     It is without loss for seller to commit to a truthful single parameter mechanism $\mathcal{M}=(x,p)$ and
     for the buyer to participate and truthfully reveal v if and only if $u^\mathcal{M}(v) = \int_0^v x(z)dz - p_0 \geq c$.
 \end{lemma}

 Next to maximize the mechanism one has to find its characteristics as well.
 In the single parameter setting, \cite{myerson} provided such as well.

 \begin{lemma}[\cite{myerson}, Myerson Revenue]
     \label{lemma:myerson-revenue}

     Let $\mathcal{M}=(x,p)$ be a truthful single-parameter mechanism, then the expected revenue equals the expected virtual welfare. That is
     \begin{align*}
         \mathbf{E}_v\left[p(v)\right]
         = \mathbf{E}_v\left[x(v)\varphi(v)\right] + p_0
     \end{align*}

     where $\varphi(t) = t - \frac{1 - F(t)}{f(t)}$
 \end{lemma}

 Following from that we derive a expression for the expected revenue for private participation costs.
 For that we first formulize the mechanism decribed in \autoref{lemma:thruthful-mechanism}.

 Given a thruthful single parameter mechanism $\mathcal{M}=(x,p)$.
 For that, we define $v_x(c)$ as the minimum value a player must have, s.t. she would still participate in the mechanism $\mathcal{M}$ when having participation cost $c$.

 \begin{align*}
     v_x(c) = \inf_{v \in \RR_{\geq 0}} \{vx(v) - p(v) \geq c\}
 \end{align*}

 Next we can define the allocation function $x_c$ as simply implementing the characterization of \autoref{lemma:thruthful-mechanism},
 by allocating $x(v)$ if the player has a value that is at least $v_x(c)$ and $0$ otherwise.

 \begin{align*}
     x_c(v) = \begin{cases}
                  x(v) & v \geq v_x(c)    \\
                  0    & \text{otherwise}
              \end{cases}
 \end{align*}

 Similar the payment rule $p_c(v)$ delegates the the payment rule $p$ in the case that $v \geq v_x(c)$ and otherwise expects no payment.
 Together, they define the mechanism $\mathcal{M}_c = (x_c, p_c)$ induced by the mechanism $\mathcal{M}$.

 If we now define a virtual value function that is conditional on $c$, by using the conditional cumulative distribution $\bar{F}_c$ and the conditional densitiy function $\hat{f}_c$ as
 $\varphi_c(v) = v - (1- \bar{F}_c(v)) / \hat{f}_c(v)$ then we can express the expected revenue in \autoref{theorem:expected-revenue}.

 \begin{theorem}[\cite{primary}]
     \label{theorem:expected-revenue}
     Given any $\bar{F}$ with marginal cost distribution $G$, any mechanism $\mathcal{M}=(x,p)$, then the expected revenue $\mathbf{E}_{t \sim \bar{F}}\left[p_c(v)\right]$ of the seller is

     \begin{align*}
         \mathbf{E}_{c \sim G}\left[\mathbf{E}_{v\sim\bar{F}_c}\left[x_c(v)\varphi_c(v)\right] - (1-\bar{F}_c(v_x(c))) \cdot \max\{-p_0,c\}\right]
     \end{align*}
 \end{theorem}

 \begin{proof}
     We proof this using a case distinction and drawing the expectation around it.
     First start by fixing $c$ arbitrarily.
     Consider the case that $c > -p_0$. Note that $-p_0$ corresponds to a payment the player would get from the mechanism for participating.
     That means in the case that $c > -p_0$, we do not simply always participate, since our utility could be negative. Therefore we also have $v_x(c) > 0$.

     We first consider the case that we partipate, i.e. $v \geq v_x(c)$.
     In that case the payment the player has to pay the payment $p(v)$ of the single parameter mechanism.

     \begin{align}
         \label{eq:payment-participating}
         p_c(v) = p(v) = v x(v) - \int_0^v x(t) dt + p_0
     \end{align}

     We use linearity of the intergral and the definition of utility to express this in terms of the allocation function $x_c$.
     \begin{align*}
         \int_0^v x(t) dt & = \int_0^{v_c(x)} x(t) dt + \int_{v_c(x)}^v x(t) dt     \\
                          & = u^\mathcal{M}(v_c(x)) + p_0 + \int_{v_c(x)}^v x(t) dt \\
                          & = c + p_0 + \int_{v_c(x)}^v x(t) dt
     \end{align*}
     Here the last step uses that $u^\mathcal{M}(v_c(x)) = c$ per definition of $v_c(x)$.

     Plugging this into \autoref{eq:payment-participating} yields
     \begin{align*}
         p_c(v) = v x_c(v) - \int_0^v x_c(t) dt - c
     \end{align*}

     Next we consider the case that the player does not partitipate, i.e. $v < v_x(c)$.
     In that case, the payment rule requests no payment, i.e. $p_c(v) = 0$, yet conveniently some part of the expression found before are also $0$:

     \begin{align*}
         p_c(v) = 0 = v x_c(v) - \int_0^v x_c(t) dt
     \end{align*}

     Thus we can represent this entire case using an inditcator function:
     \begin{align*}
         p_c(v) = 0 = v x_c(v) - \int_0^v x_c(t) dt - c \1{v \geq v_x(c)}
     \end{align*}

     Drawing the expectation around the value $v \sim \bar{F}_c$ while keeping the participation cost still fixed yields

     \begin{align*}
         \Ex[v \sim \bar{F}_c]{p_c(v)} & = \Ex[v \sim \bar{F}_c]{v x_c(v) - \int_0^v x_c(t) dt - c \1{v \geq v_x(c)}}   \\
                                       & = \Ex[v \sim \bar{F}_c]{v x_c(v) - \int_0^v x_c(t) dt} - \Pr{v \geq v_x(c)}c   \\
                                       & = \Ex[v \sim \bar{F}_c]{x_c(v) \varphi_c(v)} - (1 - \bar{F}_c(v_x(c))) \cdot c \\
     \end{align*}

     Here in the last step we used that for fixed $c$, $M_c$ is a thruthful mechanism thus we can apply \autoref{lemma:myerson-revenue} and we also used the cumulative distribution.

     Now we consider the opposite case, that is that $c \leq -p_0$.
     Analogously to the case before, $c \leq -p_0$ means that the mechanism's payment for simply participating already pays of our participation cost.
     Certainly in that case $v_x(c) = 0$.

     Therefore the player always participates and pays $p(v)$. In the case that $v_x(c) = 0$, we have $x_c(v) = x(v)$ for all $v \in \RR_{\geq 0}$, therefore:
     \begin{align*}
         p_c(v) = p(v) = v x(v) - \int_0^v x(t)dt + p_0 = v x_c(v) - \int_0^v x_c(t)dt + p_0
     \end{align*}

     Drawing again the expectation around the value $v \sim \bar{F}_c$ while keeping the participation cost still fixed yields


     \begin{align*}
         \Ex[v \sim \bar{F}_c]{p_c(v)} & = \Ex[v \sim \bar{F}_c]{v x_c(v) - \int_0^v x_c(t)dt + p_0}                  \\
                                       & = \Ex[v \sim \bar{F}_c]{x_c(v) \varphi_c(v)} + p_0                           \\
                                       & = \Ex[v \sim \bar{F}_c]{x_c(v) \varphi_c(v)} + (1 - \bar{F}_{c}(v_x(c))) p_0
     \end{align*}

     Where in the last equality, we used that since $v_x(c) = 0$, $\bar{F}_{c}(v_x(c)) = 0$ as well.

     Now we can finally draw the expectation around the previously fixed $c$ and get to the wished theorem:

     \begin{align*}
         \Ex{p_c(v)} & = \Ex[c \sim G]{\Ex[v \sim \bar{F}_c]{x_c(v) \varphi_c(v)} + (1 - \bar{F}_{c}(v_x(c))) \max\{c, -p_0\} }
     \end{align*}
 \end{proof}

\end{section}

\bibliographystyle{plain}
\bibliography{references} % Insert name of .bib file

\end{document}
